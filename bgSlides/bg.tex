\documentclass[09pt]{beamer}

\usetheme{Szeged}
\usecolortheme{rose}

\usepackage{color}
\usepackage{beamerthemeshadow}
\usepackage{mathrsfs}
\usepackage{url}
\usepackage{listings}

\definecolor{greenish}{RGB}{152,204,112}
\definecolor{redish}{RGB}{244,158,196}

\setbeamertemplate{navigation symbols}{}
\newcommand*{\LargerCdot}{\raisebox{-0.25ex}{\scalebox{3.0}{$\cdot$}}}
\expandafter\def\expandafter\insertshorttitle\expandafter{%
  \insertshorttitle\hfill%
  \insertframenumber}%\,/\,\inserttotalframenumber}


\begin{document}

\author{Kacper Sokol}
\title[Learning \texttt{Prolog} rules]{Learning \texttt{Prolog} rules \& \\extracting features from signals}
\institute{Department of Computer Science}
\titlegraphic{\includegraphics[scale=.5]{../paper/gfx/UOB-logo}}
\date{\today}

\begin{frame}
\titlepage
\end{frame}

\begin{frame}
  \frametitle{Table of contents}
  \tableofcontents
\end{frame} 


\section{Background}

  \subsection{World's population}
  \begin{frame}[plain]
    \frametitle{Population ageing}
    \begin{figure}
      \centering
      \includegraphics[scale=.4]{gfx/populationOver65}
      \caption{Percentage of world population over 65, 1950--2050; \cite{populationAgeing}}
    \end{figure}
  \end{frame}

  \subsection{SPHERE Project}
  \begin{frame}[allowframebreaks]
    \frametitle{SPHERE project~\cite{sphere}:\\An EPSRC Interdisciplinary Research Collaboration} % (IRC)
    \begin{block}{The Challenge}
      \begin{columns}
        \begin{column}{6cm}
          \begin{itemize}
            \item Global population health issues
            \item Ageing population
            \item Increasing healthcare costs
            \item Decreasing quality of life
          \end{itemize}
        \end{column}
        \begin{column}{4cm}
          \begin{figure}
            \includegraphics[scale=.13]{gfx/sphere} 
          \end{figure}
          \end{column}
      \end{columns}
    \end{block}
    \pause
    \begin{block}{The Technology}
      \begin{itemize}
        \item Sensors development for smart house applications
        \item \textbf{Use acquired information} to identify medical or well-being issues: predict falls, detect strokes, analyse eating behaviour, and detect periods of depression or anxiety
      \end{itemize}
    \end{block}
    \pause
    \begin{block}{The Approach}
      Collaboration of clinicians, engineers, designers and social care professionals as well as members of the public to develop helpful technologies:
      \begin{itemize}
        \item Focus on real-world technologies acceptable in people's homes
        \item Address real healthcare problems in a cost effective way
      \end{itemize}
    \end{block}
  \end{frame}



\section{Applications}

\subsection{Spatio-temporal data}
\begin{frame}[fragile]
\frametitle{...Use acquired information...}
{\tiny
\begin{columns}
\begin{column}{5.4cm}
\begin{example}[\tiny Smart house output~\cite{cook2009assessing}]
\begin{figure}
\lstset{
captionpos=b,
% frame=single,
language=HTML,
breaklines=true,
% caption=CASAS dataset structure,
% label=lst:data,
float=tb,
basicstyle=\tiny
}
\begin{lstlisting}
...
2008-02-26 10:52:58.577436 M17   OFF
2008-02-26 10:52:59.792264 M17   ON
...
2008-02-26 10:53:43.512642 I02   ABSENT
2008-02-26 10:53:43.978491 I01   ABSENT
...
2008-02-26 10:53:52.112690 AD1-B 0.0421
2008-02-26 10:53:54.721822 M17   ON
...
\end{lstlisting}
\caption{\tiny CASAS experiment dataset structure\label{lst:data}}
\end{figure}
\end{example}
\end{column}
\pause
\begin{column}{5.4cm}
\begin{example}[\tiny Data transformation]
\begin{figure}
\lstset{
captionpos=b,
% frame=single,
language=HTML,
breaklines=true,
% caption=Learnt rules,
% label=lst:rules,
float=tb,
basicstyle=\tiny
}
\begin{lstlisting}
...
sensor(m09, false, relative, 4772648).
sensor(m09, false, absolute, 120411621).
sensor(m09, false, sequence, 9).
sensor(m09, false, windowed, 0).

sensor(i08, false, relative, 5692364).
sensor(i08, false, absolute, 120411622).
sensor(i08, false, sequence, 10).
sensor(i08, false, windowed, 1).
...
\end{lstlisting}
\tiny\caption{\tiny Learnt rules\label{lst:rules}}
\end{figure}
\end{example}
\end{column}
\end{columns}
}
\end{frame}

\subsection{Goal}
\begin{frame}
  \frametitle{Extract knowledge}
  \begin{block}{Objectives}
    \begin{itemize}
      \item Learn activity recognition model
      \item Distinguish multiple occupiers
      \item Extract more informative signal features
    \end{itemize}
  \end{block}
  \pause
  \begin{block}{Challenges}
    \begin{itemize}
      \item Lack of labelled data to train and test models
      \item Incompleteness of data
      \item Noisy data
      \item Time handling (unbounded variable)
      \item Real valued sensor output
      \item Information loss: room and sensor layout
    \end{itemize}
  \end{block}
\end{frame}

\begin{frame}[fragile]
\frametitle{Output}
\begin{example}[Rules for activity recognition]
\begin{figure}
\lstset{
captionpos=b,
% frame=single,
language=HTML,
breaklines=true,
% caption=Learnt rules,
% label=lst:rules,
float=tb
}
\begin{lstlisting}
activity(Person, cooking, TimeWindow) :-
    location(Person, TimeWindow, kitchen),
    device(TimeWindow, hob).

location(Person, TimeWindow, kitchen) :-
    sensor(m08, on, absolute, TimeWindow),
    sensor(m09, on, absolute, TimeWindow).

device(TimeWindow, hob) :-
    sensor(ad1-a, on, absolute, TimeWindow).
\end{lstlisting}
\caption{Learnt rules\label{lst:rules}}
\end{figure}
\end{example}
\end{frame}  


\section{Techniques}

  \subsection{Inductive Logic Programming}
  \begin{frame}%[plain]
    \frametitle{Inductive Logic Programming (ILP)}
    \begin{block}{ILP~\cite{muggleton1994inductive,muggleton1995inverse}}
      \begin{itemize}
        \item Fusion of \emph{inductive learning} and \emph{logic programming} aiming at best of both worlds
        \item Powerful knowledge representation as \emph{first-order logical rules}
        \item Build model from available observations
        \item Generate new knowledge form experience
        \item Background knowledge incorporated into model
        \item Induction as a basic mode of inference: generalization of specific observations to theories
        \item Human readable model
      \end{itemize}
    \end{block}
  \end{frame}

  \begin{frame}%[plain]
    \frametitle{Power of ILP}
    \begin{block}{How does it work?}
        \begin{figure}
          \centering
          \includegraphics[scale=.5]{../paper/gfx/ilp}
          \caption{ILP scheme\label{fig:ilp}}
        \end{figure}
    \end{block}
  \end{frame}

  \subsection{ILP in healthcare}
  \begin{frame}
    \frametitle{Advantages}
    \begin{block}{Innovative?}
      \begin{itemize}
        \item Background knowledge: room and sensor layout, activity structure, etc.\
        \item First order logic---parametrised rules: more powerful than widely used propositional logic
        \item Human readable models easy to inspect and tune
      \end{itemize}
    \end{block}
    \pause
    \begin{block}{Suitable?}
      \begin{itemize}
        \item Use of background knowledge not contained in the dataset gives more power
        \item Customisable recognition model
        \item Possible to extend with \emph{Data Narrative}
      \end{itemize}
    \end{block}
  \end{frame}

\section{Contribution}

  \subsection{Challenges}
  \begin{frame}
    \frametitle{Challenges}
    \begin{itemize}
      \item Design customisable spatio-temporal (smart house) data generator to test and design models
      \item Investigate spatio-temporal data representation as logical facts
      \item Research representation of unbounded variables like time in first-order logic
      \item Examine use of background knowledge not contained in the original dataset
      \item Create and test different activity recognition models in ILP
      \item Build mechanism for new features discovery
    \end{itemize}
  \end{frame}

  \subsection{Deliverables}
  \begin{frame}%[plain]
    \frametitle{Deliverables}
    \begin{itemize}
      \item Easy to use spatio-temporal data generator
      \item Framework for spatio-temporal data analysis with ILP
      \item Limitations of ILP when applied to spatio-temporal data
      \item Methodology for time representation in knowledge form
      \item Evaluation of ILP model: contrast \& compare against widely used solutions
      \item Feature extraction mechanism
    \end{itemize}
  \end{frame}




\section*{}%References
  \begin{frame}[allowframebreaks,plain]
    \frametitle{References}
    \bibliographystyle{amsalpha}
    \bibliography{../paper/yhpargoil.bib,bg.bib}
  \end{frame}

  \begin{frame}[plain]
    \centering \Huge Q\&A \par
  \end{frame}

\end{document}
